\documentclass[12pt, a4paper]{article}

\usepackage{german}
\usepackage[utf8]{inputenc}
\usepackage[T1]{fontenc}
\usepackage{amsmath}
\usepackage{amssymb}
\usepackage{hyperref}
\usepackage{marvosym}

\newcommand{\changefont}[3]{
\fontfamily{#1} \fontseries{#2} \fontshape{#3} \selectfont}

\usepackage{tikz}
\usetikzlibrary{trees}

\usepackage{bigstrut}

\title{KI - Praktikum2}
\author{Gruppe A\_Blau\_WS1415\\\\
    Felix Gebauer\\
    Steffen Lang\\
    Mara Braun\\
    Christoph Hegemann\\
    Janis Saritzoglou}


\begin{document}
\maketitle
\newpage
\section{}
Gene:
\begin{itemize}
\item A: 500
\item B: 10
\item C: 0
\item D: -200
\end{itemize}
8 Gene pro Individuum. Startpopulation 500 zufallsgenerierte
Individuen.

\subsection*{a}
Sei $\mathbb{G} = \{500,10,0,-200\}$ die Menge aller Gene, dann ist
$\mathbb{I} = \mathbb{G}^8$ die Menge aller möglichen Individuen mit 8
Genen.

\subsubsection*{Fitnessfunktion}
Sei $P \subseteq \mathbb{I}$ eine Population.\\
Dann weist die Fitnessfunktion $F$ einem jeden Individuum $i \in P$
einen Wert zu.\\
\begin{align*}
F: \mathbb{I} \longrightarrow \mathbb{Z}\\
F(i) = \sum_j i_j
\end{align*}

\end{document}

%%% Local Variables:
%%% mode: latex
%%% TeX-master: t
%%% End:
